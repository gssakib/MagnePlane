The traffic between the notional end points and the market demand for tickets at a given cost will determine the amount of passengers the Hyperloop must be able to accommodate in a given period of time. The number of passengers per unit of time that the Hyperloop can transport is equal to the number of passengers per pod times the frequency at which pods depart. If a constant pod capacity is assumed, then the only way to accommodate for periods of high demand is to increase the frequency at which pods depart. It is important to understand the limitations of pod frequency in order to obtain a reasonable sense of maximum passenger throughput.
There are three factors that must be considered when evaluating limitations on pod frequency: safety, the amount of time it takes to board, and the number of pods available. Pods must be spaced out such that each pod can slow down and stop before hitting the pod in front of it in the event of an emergency stop. Simple linear acceleration equations can be used to determine the minimum distance separating to allow for deceleration and are shown below
\begin{equation}
	\label{eq:linear_acceleration}
	Insert Linear Acceleration Equations
\end{equation}
\end{equation}
Assuming a braking deceleration of 1g, a Mach number of .8, and a tunnel temperature of 320 K \cite{Chin}, the minimum distance between pods is calculated to be 4.2 km with no safety margin. At top speed, the deceleration time is calculated to be 29.2 s. This means that, with no safety margin, the maximum pod frequency is about 2 pods per minute. This frequency should easily be satisfied, even with a safety margin, due to the time that must be allowed for passengers to board. It is likely that passengers will need considerably more than 30 seconds to board, meaning that safety is not likely to be the the factor limiting maximum pod frequency. Finally, in this analysis, it is assumed that the operator must have enough pods to fill the entire tube when launching flights at a given frequency. Consequently, having less time in between pods reduces the distance in between pods, which increases the number of pods required to fill the tube and sustain the desired pod frequency. Due to these three limiting factors, lower pod frequencies are desirable because they increase the safety margin in the event of emergency braking, increase the time passengers have to board, and reduce the number of pods that the operator must have in order to sustain a given pod frequency. The effects that changes in pod capacity will have on system design and performance will be discussed in further detail later in this study.
